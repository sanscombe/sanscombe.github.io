%!TEX spellcheck = English (United States) [w_accents]
%Paul Shafer 03 May 2023
\documentclass[11pt]{amsart}
\usepackage{amsmath,amssymb,amsfonts,amsthm,amsrefs}
\usepackage[margin=1in]{geometry}
\usepackage{hyperref}

\title{Complexity of descriptions and relations in computable structure theory}
\author{Ekaterina Fokina}
\newcommand{\Nb}{\mathbb{N}}

\newcommand{\leqT}{\leq_\mathrm{T}}

\newcommand{\mc}[1]{\mathcal{#1}}

\begin{document}

\maketitle

\section*{Abstract}

Abstract: In this talk we consider structures given in an effective way. For example, a graph is computable if its set of vertices is a computable set and the edge relation is computable. Given such a computable structure, how hard is it to determine if this structure has particular algebraic, model-theoretic or algorithmic properties? For example, how hard is it to find out that a binary relation on a set is actually a linear order? Or to say that a structure is a prime model of its theory, etc.? In this talk we will discuss one of the approaches to investigate this kind of questions. We will then extend this approach to study the classification problem for computable structures. We will look at the question, how hard it is to determine whether two computable structures are isomorphic (or otherwise equivalent, for a suitable interesting equivalence relation). We will also see how to go beyond computable structures and use the same ideas to characterise and classify classes of structures given in a computably enumerable (but not necessarily computable) way.
\end{document}