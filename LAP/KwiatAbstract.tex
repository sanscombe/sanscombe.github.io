%!TEX spellcheck = English (United States) [w_accents]
%Paul Shafer 03 May 2023
\documentclass[11pt]{amsart}
\usepackage{amsmath,amssymb,amsfonts,amsthm,amsrefs}
\usepackage[margin=1in]{geometry}
\usepackage{hyperref}

\title{Introduction to continuum theory and projective Fra\"{i}ss\'e theory}
\author{Aleksandra Kwiatkowska}

\newcommand{\Nb}{\mathbb{N}}

\newcommand{\leqT}{\leq_\mathrm{T}}

\newcommand{\mc}[1]{\mathcal{#1}}

\begin{document}

\maketitle
\section*{Abstract}



A continuum is a compact connected (separable) space. I will first discuss a number of properties (such as indecomposability, homogeneity) and examples of 1-dimensional continua. These are solenoids, Knaster continua, Menger universal curve, pseudo-arc, and many more.

Then I will present the projective Fra\"{i}ss\'e limit construction introduced by Irwin and Solecki. They applied it to give a new construction of the pseudo-arc and to show projective homogeneity and projective universality of that space. Since then this construction found applications in studying many other continua and their homeomorphism groups.
\end{document}