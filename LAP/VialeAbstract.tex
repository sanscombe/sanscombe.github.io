%!TEX spellcheck = English (United States) [w_accents]
%Paul Shafer 03 May 2023
\documentclass[11pt]{amsart}
\usepackage{amsmath,amssymb,amsfonts,amsthm,amsrefs}
\usepackage[margin=1in]{geometry}
\usepackage{hyperref}

\title{What model companionship can say about the continuum problem}
\author{Matteo Viale}

\newcommand{\Nb}{\mathbb{N}}

\newcommand{\leqT}{\leq_\mathrm{T}}

\newcommand{\mc}[1]{\mathcal{#1}}

\begin{document}

\maketitle

\section*{Abstract}
We present recent results on the model companions of set theory. 

We start by describing the dependence of the notion of model companionship on the signature, and then we analyze this dependence in the specific case of set theory.  We argue that the most natural model companions of set theory describe (as the signature in which we axiomatize set theory varies) theories of $H_{\kappa^+}$, as $\kappa$ ranges among the infinite cardinals. We also single out $2^{\aleph_0}=\aleph_2$ as the unique solution of the continuum problem which can (and does) belong to some model companion of set theory (enriched with large cardinal axioms). 

While doing so we bring to light that set theory enriched by large cardinal axioms in the range of supercompactness has as its model companion (with respect to its first order axiomatization in certain natural signatures) the theory of $H_{\aleph_2}$ as given by a strong form of Woodin's axiom $(*)$ (which holds assuming $\mathsf{MM}^{++}$).\end{document}