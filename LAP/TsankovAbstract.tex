%!TEX spellcheck = English (United States) [w_accents]
%Paul Shafer 03 May 2023
\documentclass[11pt]{amsart}
\usepackage{amsmath,amssymb,amsfonts,amsthm,amsrefs}
\usepackage[margin=1in]{geometry}
\usepackage{hyperref}

\title{Infinitary continuous logic and applications to descriptive set theory}
\author{Todor Tsankov}
\address{}
\email{}
\urladdr{}

\newcommand{\Nb}{\mathbb{N}}

\newcommand{\leqT}{\leq_\mathrm{T}}

\newcommand{\mc}[1]{\mathcal{#1}}

\begin{document}

\maketitle

\section*{Abstract}


The logic $\mathcal L_{\omega_1\omega}$ is similar to classical first-order logic, except that it allows for countably infinite conjunctions and disjunctions. This increases the expressive power of the logic but at the cost of losing the compactness theorem, which is the basis of much of modern model theory. However, the countable connectives play well with Borel sets and this logic has found a number of applications in descriptive set theory. I will discuss the appropriate version of $\mathcal L_{\omega_1\omega}$ for continuous logic and some descriptive-set theoretic applications. The talk is based on two papers, the first joint with Ita\"i Ben Yaacov, Michal Doucha and Andr\'e Nies, and the second joint with Andreas Hallb\"ack and Maciej Malicki.

\end{document}