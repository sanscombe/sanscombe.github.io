%!TEX spellcheck = English (United States) [w_accents]
%Paul Shafer 03 May 2023
\documentclass[11pt]{amsart}
\usepackage{amsmath,amssymb,amsfonts,amsthm,amsrefs}
\usepackage[margin=1in]{geometry}
\usepackage{hyperref}

\title{The quantifier $Q$ "there exist uncountably many", quasi-minimality, and complex exponentiation}
\author{Vincenzo Mantova}


\newcommand{\Nb}{\mathbb{N}}

\newcommand{\leqT}{\leq_\mathrm{T}}

\newcommand{\mc}[1]{\mathcal{#1}}

\begin{document}

\maketitle

\section*{Abstract}

In classical model theory, infinite definable sets can have any arbitrary size. But what if we want countable sets to stay put? We will go back to 1970 and Keisler's four axioms for the quantifier $Q$ ``there exist uncountably many", which can be used to pin down the cardinality of our preferred definable sets. The proof of completeness of these axioms prompts us to go further: instead of just pinning down the cardinality of countable sets, we should pin down the sets themselves; more precisely, we should only look at elementary extension where the countable definable sets do not grow at all (just like finite definable sets do not grow when going to elementary extensions). Fast forward to this century, we will see how this approach plays out in Zilber's study of complex exponentiation and his conjecture that complex exponentiation is quasi-minimal, that is to say, that every definable subset of the complex numbers is either countable or the complement of a countable set.

\end{document}